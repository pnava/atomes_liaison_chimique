%--------------------------------------------------------------------
\titreTD{\thenumTD}{VSEPR et introduction \`a l'hybridation}

\exo{Mol\'ecules simples}

Dans un tableau, pour chaque mol\'ecule de l'ensemble I, pr\'ecisez pour chaque atome diff\'erent de H~:

\begin{itemize}
\item l'expression VSEPR AX$_n$E$_m$;
\item la figure de r\'epulsion;
\item la g\'eom\'etrie r\'eelle;
\item le type d'hybridation.
\end{itemize}


\exo{Mol\'ecules organiques simples}

Pour les mol\'ecules de l'ensemble II, \'etablir le type d'hybridation de chaque atome de carbone.

\exo{R\'ecapitulatif}

Compl\'eter le tableau suivant en indiquant l'expression VSEPR AX$_n$E$_m$, le type 
d'hybridation et le degr\'e d'oxydation des atomes de carbone en gras.

\clearpage

%\rotatebox{90}{%
%\parbox{15cm}{
\begin{landscape}
\begin{center}
%\begin{tabular*}{1.0\textheight}{@{\extracolsep{\fill}}rlllll}
\renewcommand{\arraystretch}{4}
\begin{tabular*}{20cm}{@{\extracolsep{\fill}}|r|c|c|c|c|c|}
\hline
                                       & \multicolumn{5}{c|}{\textbf{R$^\mathbf{1}$}} \\
\hline
                                       &CH$_3$ & OH & OCH$_3$ & NH$_2$ & NHCH$_3$ \\
\hline
\textbf{C}H$_3$-R$^1$                  &       &    &         &        & \\\hline 
%
CH$_3$-\textbf{C}H$_2$-R$^1$           &&&&&\\\hline

\begin{pspicture}(2.3,1)                     
%\showgrid
\put(0.0,0.6){CH$_3$}\psline(0.8,0.6)(0.97,0.4)
\put(1.0,0.3){\textbf{C}H-R$^1$}
\put(0.0,0.0){CH$_3$}\psline(0.8,0.15)(0.97,0.4)
\end{pspicture}                        &&&&&\\\hline
%
%\begin{pspicture}(1.9,1.5)                     
%%\showgrid
%\put(0.0,0.6){CH$_3$-\textbf{C}-R$^1$}
%\put(0.85,0.0){CH$_3$}\psline(1.05,0.35)(1.05,0.57)
%\put(0.85,1.15){CH$_3$}\psline(1.05,0.90)(1.05,1.12)
%\end{pspicture}                        &&&&&\\\hline
%
CH$_3$-\textbf{C}(O)-R$^1$ &&&&&\\\hline
%H\textbf{C}O-R$^1$       &&&&&\\\hline
R$^1$-\textbf{C}HO         &&&&&\\\hline
\end{tabular*}
\end{center}
\end{landscape}
%}
%}
