\titreTD{\thenumTD}{Mod\`ele de Bohr --  Notions de spectroscopies}

%\exo{L'atome d'hydrog\`ene}
%
%Le mod\`ele de l'atome d'hydrog\`ene propos\'e par Rutherford et Perrin est le suivant~:
%un \'electron de charge $e$ tourne autour d'un proton en parcourant, d'un mouvement
%uniforme, une orbite circulaire de rayon $r$.
%
%\begin{enumerate}[\bf 1)]
%
%\item On se propose de trouver l'expression de l'\'energie totale de l'interaction \'electron/proton
%en employant les lois de la m\'ecanique classique et la loi de Coulomb.
%\begin{enumerate}
%\item Exprimer les forces en pr\'esence au niveau de l'\'electron, puis en faisant le bilan de ces forces
%\`a l'\'equilibre montrer que~:
%
%\begin{equation}
%\label{eq_bohr1}
%mv^2 = \frac{e^2}{4 \pi \varepsilon_0 r}
%\end{equation}
%
%\item En consid\'erant l'\'energie comme la somme des \'energies cin\'etique et potentielle,
%puis en se servant de l'\'equation~\ref{eq_bohr1}, montrer que~:
%
%\begin{equation}
%\label{eq_bohr2}
%E = \frac{-e^2}{8 \pi \varepsilon_0 r}
%\end{equation}
%
%\end{enumerate}
%
%\item En  posant l'hypoth\`ese de Bohr~:
%
%\begin{equation}
%\label{eq_bohr3}
%mvr = n\frac{h}{2\pi}
%\end{equation}
%
%\'etablir l'expression de $r$ dans l'atome d'hydrog\`ene, puis la nouvelle expression de l'\'energie.
%
%\item  Calculer le rayon de la premi\`ere orbite de Bohr (n=1) pour l'atome d'hydrog\`ene,
%en v\'erifiant l'\'equation aux dimensions.
%
%%\item Calculer la vitesse (en ms$^{-1}$) de l'\'electron sur la premi\`ere orbite de Bohr de l'atome
%%d'hydrog\`ene.
%\end{enumerate}


\exo{Spectroscopies d'absorption et d'emission: c'est lumineux.}
Pour attirer un partenaire sexuel, les vers luisant produisent une lumière verte.
Cette lumière est due à une réaction chimique:
\footnote{Bulletin de l'Académie Lorraine des Sciences 2005, 44 (1-4)}
il s’agit de l'oxydation d'une molécule organique, la luciférine qui s'oxyde sous l’effet de l’ATP
et d’ions magnésium, catalysée par une enzyme, la luciférase.
L’énergie biochimique se transforme en lumière.
\footnote{adapté de Alice Despinoy – http://lanaturedepres.fr}
%\begin{enumerate}[\bf 1)]
	%\item Est-ce un phénomène d'absorption ou d'emission?
	Est-ce un phénomène d'absorption ou d'emission?
%\end{enumerate}

\exo{Spectroscopies d'absorption et d'emission: Bon appétit bien sûr!}
Les cochenilles sont de charmants insectes d'environ 5mm.
Elles se défendent en sécrétant de l'acide carminique qui est utilisé depuis le Moyen Âge comme colorant
rouge comme son nom l'indique.
Actuellement, ce colorant est utilisé comme colorant alimentaire (sous le nom de E120 dans des viandes, sirops, biscuits, bonbons, etc.), cosmétique (où il prend le nom CI 75470 dans des dentifrices, rouges à levre, etc) ou autre.

%\begin{enumerate}[\bf 1)]
	%\item Lorsque l'on regarde un verre de grenadine, il paraît rouge.
	%	Est-ce un phénomène d'absorption ou d'emission?
	      Lorsque l'on regarde un verre de grenadine, il paraît rouge.
		Est-ce un phénomène d'absorption ou d'emission?
%	\item Les états fondamentaux et excités impliqués sont-ils atomiques ou moléculaires?
%\end{enumerate}

\exo{Notions de spectroscopies}

\begin{enumerate}[\bf 1)]
\item On consid\`ere l'atome d'hydrog\`ene (Z=1) et d'h\'elium (Z=2).
\begin{enumerate}
\item Expliquer pourquoi He$^+$ est un hydrog\'eno\"ide.
%\item Calculer en eV les potentiels d'ionisations pour H et He$^+$.
\item Calculer les \'energies des diff\'erents niveaux quantiques ($n$ = 2 \`a 6) pour l'hydrog\`ene et He$^+$
en eV et en Joules. Tracer ces niveaux sous formes de diagrammes en eV.
\end{enumerate}

\item Lorsque l'\'electron se d\'esexcite (de $n_i$ \`a $n_f < n_i$) il \'emet de la lumi\`ere avec
une \'energie correspondant exactement \`a la diff\'erence d'\'energie entre les niveaux $n_f$ et $n_i$.
On rappelle que $E=hc/\lambda$. Le spectre d'\'emission de la s\'erie de Balmer pour l'hydrog\`ene
a \'et\'e le premier observ\'e car il est dans le domaine spectral visible (400 \`a 800 nm).
\begin{enumerate}
\item Montrer que la s\'erie de Lyman ($n_f =1$) n'est pas dans le domaine visible.
\item A quelle \'energie (en eV) correspond la longueur d'onde 400 nm ?
\item A quelle \'energie (en eV) correspond la longueur d'onde 800 nm ?
\item Sur la base de l'exercice 1c), donner pour He$^+$ des valeurs de $n_i$ et de $n_f$ pour
que la transition se produise dans le visible, commenter.
\end{enumerate}

\item On considère à présent l’atome d’hydrogène dans l’état électronique excité caractérisé par
n$_2$ = 3.
\begin{itemize}
\item Quels sont les chemins de désexcitation possibles pour que l’atome arrive à son niveau
fondamental n 1 = 1 ?
\item Combien de raies d’émission observe-t-on ? Lesquelles ?
\item Mêmes questions si l’atome d’hydrogène est dans l’état excité caractérisé par n$_2$ = 4.
\item L’atome H étant toujours dans l’état excité n$_2$ = 4, on suppose qu’au cours de la
désexcitation, la population se répartit de façon égale entre les différents niveaux
possibles. Par exemple si l’on a 100 atomes excités au départ, et s’il y a 3 niveaux
accessibles lorsqu’un électron fait une transition vers un niveau inférieur, alors 33
électrons feront la transition vers chacun des niveaux possibles. Et le même
raisonnement se poursuit pour les transitions qui finissent par ramener les atomes à
leur niveau fondamental. Combien d’atomes feront alors la transition 2$\rightarrow$1 sachant que
l’on a 100 atomes excités au départ ?
\end{itemize}


%\item Bien avant la th\'eorie de Bohr, Balmer et Rydberg ont \'etabli la relation empirique
%permettant de calculer les longueurs d'onde des raies d'\'emission du spectre de l'atome d'hydrog\`ene~:
%\begin{equation}
%\label{eq_ryd}
%\frac{1}{\lambda} = R_H \left( \frac{1}{n^2_f} - \frac{1}{n^2_i} \right)
%\end{equation}
%dans laquelle $n_f$ et $n_i$ sont des entiers et $R_H$ une constante, appel\'ee constante de Rydberg.
%Trouver que $R_H = 1,097.10^7 m^{-1}$.
%%Calculer la valeur de la constante de Rydberg.
\end{enumerate}

%{\textcolor{red} Donn\'ees}

%\vspace{1cm}
%\small{
%\begin{tabular}{lll}
%\multicolumn{3}{l}{\bf Donn\'ees}\\
%$c$              & (vitesse de la lumi\`ere) & = 3*108 m.s$^{-1}$\\
%$h$              & (constante de Planck)     & = 6.624*10$^{-34}$ J.s\\
%$m$              & (masse de l'\'electron)   & = 9.10*10$^{-31}$ kg\\
%$e$              & (charge \'el\'ementaire)  & = 1.602*10$^{-19}$ C\\
%$\varepsilon_0$  & (permitivit\'e du vide)   & = 8.854*10$^{-12}$ C$^2$.J$^{-1}$.m$^{-1}$\\
%\multicolumn{3}{l}{1 J = 1 m$^2$kg.s$^{-2}$}\\
%\multicolumn{3}{l}{1 J = 6.241*10$^{18}$ eV}\\
%\end{tabular}
%}
\exo{Spectroscopies d'absorption et d'emission: Trop chaud!}
Le Soleil emet une lumière à large spectre, c'est-à-dire que
l'on peut faire l'hypothèse que toutes les longueurs d'onde sont émise
avec une égale intensité.
Or, l'analyse du spectre de la lumière solaire à la surface de la Terre met en évidence
des longueurs d'onde manquantes que l'on appelle raies de Fraunhofer :\\
\begin{pspicture}(\textwidth,5cm)
%	\put(0,5){\psspectrum[lines={299,302,336,358,382,393,396,410,430,430,434,438,466,486,495,516,516,516,517,518,527,546,587,588,589,627,656,686,759,822,898},lwidth=0.04,axe,absorption,brightness=1.0](\linewidth,2.5)}
%	\put(0,2){\psspectrum[lines={410,434,486,656},lwidth=0.04,axe,absorption,brightness=1.0](\linewidth,2.5)}
	\put(0,2){\psspectrum[lines={299,302,336,358,382,393,396,410,430,430,434,438,466,486,495,516,516,516,517,518,527,546,587,588,589,627,656,686,759,822,898},lwidth=0.04,axe,absorption,brightness=1.0](\linewidth,2.5)}
%	\put(0,2){\psspectrum[lines={410,434,486,656},lwidth=0.04,axe,absorption,brightness=1.0](\linewidth,2.5)}
	\put(3,1){\textbf{Domaine visible - Longueur d'onde (nm)}}
\end{pspicture}
\begin{enumerate}[\bf 1)]
	\item Faire un schema qui fasse apparaître le Soleil, la Terre entourée de son atmosphère
		ainsi que le rayonnement solaire qui arrive sur Terre.
	\item Ces raies sont-elles dues à des phénomènes d'absorption ou d'emission?
		Expliquer à l'aide d'un schema.
	\item On note particulièrement que les 4 longueurs d'ondes suivantes sont manquantes:
		656 nm, 486 nm, 434 nm et 410 nm. Pouvez-vous proposer une explication?
%	\item À quel endroit les phénomènes responsables des raies de Fraunhofer ont-ils lieu? Est-ce
%		à la surface du Soleil, entre le Soleil et la Terre, dans l'atmosphère, à la surface
%		de la Terre?

\end{enumerate}


