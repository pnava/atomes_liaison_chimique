\titreTD{\thenumTD}{Mod\`ele de Bohr --  Notions de spectroscopies}

%\exo{L'atome d'hydrog\`ene}
%
%Le mod\`ele de l'atome d'hydrog\`ene propos\'e par Rutherford et Perrin est le suivant~:
%un \'electron de charge $e$ tourne autour d'un proton en parcourant, d'un mouvement
%uniforme, une orbite circulaire de rayon $r$.
%
%\begin{enumerate}[\bf 1)]
%
%\item On se propose de trouver l'expression de l'\'energie totale de l'interaction \'electron/proton
%en employant les lois de la m\'ecanique classique et la loi de Coulomb.
%\begin{enumerate}
%\item Exprimer les forces en pr\'esence au niveau de l'\'electron, puis en faisant le bilan de ces forces
%\`a l'\'equilibre montrer que~:
%
%\begin{equation}
%\label{eq_bohr1}
%mv^2 = \frac{e^2}{4 \pi \varepsilon_0 r}
%\end{equation}
%
%\item En consid\'erant l'\'energie comme la somme des \'energies cin\'etique et potentielle,
%puis en se servant de l'\'equation~\ref{eq_bohr1}, montrer que~:
%
%\begin{equation}
%\label{eq_bohr2}
%E = \frac{-e^2}{8 \pi \varepsilon_0 r}
%\end{equation}
%
%\end{enumerate}
%
%\item En  posant l'hypoth\`ese de Bohr~:
%
%\begin{equation}
%\label{eq_bohr3}
%mvr = n\frac{h}{2\pi}
%\end{equation}
%
%\'etablir l'expression de $r$ dans l'atome d'hydrog\`ene, puis la nouvelle expression de l'\'energie.
%
%\item  Calculer le rayon de la premi\`ere orbite de Bohr (n=1) pour l'atome d'hydrog\`ene,
%en v\'erifiant l'\'equation aux dimensions.
%
%%\item Calculer la vitesse (en ms$^{-1}$) de l'\'electron sur la premi\`ere orbite de Bohr de l'atome
%%d'hydrog\`ene.
%\end{enumerate}


\exo{Spectroscopies d'absorption et d'emission: un bond en avant.}
Dans un article intitulé "Naturally occurring fluorescence in frogs"\footnote{Proceedings of the National Academy of Sciences Mar 2017, 201701053; DOI: 10.1073/pnas.1701053114}, une équipe de
chercheurs sud-américains a mis en évidence que des grenouilles étaient fluorescentes.
\begin{enumerate}[\bf 1)]
	\item Est-ce un phénomène d'absorption ou d'emission?
	\item Les états fondamentaux et excités impliqués sont-ils atomiques ou moléculaires?
\end{enumerate}

\exo{Spectroscopies d'absorption et d'emission: Bon appétit bien sûr!}
Les cochenilles sont de charmants insectes d'environ 5mm.
Elles se défendent en sécrétant de l'acide carminique qui est utilisé depuis le moyen âge comme colorant
rouge comme son nom l'indique.
Actuellement, ce colorant est utilisé comme colorant alimentaire (sous le nome de E120 dans des viandes, sirops, biscuits, bonbons, etc.) ,cosmétique (où il prend le nom CI 75470 dans des dentifrices, rouges à levre, etc) ou autre.

\begin{enumerate}[\bf 1)]
	\item Lorsque l'on regarde un verre de grenadine, il paraît rouge.
		Est-ce un phénomène d'absorption ou d'emission?
	\item Les états fondamentaux et excités impliqués sont-ils atomiques ou moléculaires?
\end{enumerate}

\exo{Spectroscopies d'absorption et d'emission: Trop chaud!}
Le Soleil emet une lumière à large spectre, c'est-à-dire que
l'on peut faire l'hypothèse que toutes les longueurs d'onde sont émise
avec une égale intensité.
Or, l'analyse du spectre de la lumière solaire à la surface de la Terre met en évidence
des longueurs d'onde manquantes (voir figure).
\begin{enumerate}[\bf 1)]
	\item Faire un schema qui fasse apparaître le Soleil, la Terre entourée de son atmosphère
		ainsi que le rayonnement solaire qui arrive sur Terre.
	\item Ces raies sont-elles dues à des phénomènes d'absorption ou d'emission?
		Expliquer à l'aide d'un schema.
	\item On note particulièrement que les 4 longueurs d'ondes suivantes sont manquantes:
		486 nm, 656 nm, 434 nm et 410 nm. Pouvez-vous proposer une explication?
	\item À quel endroit les phénomènes responsables des raies de Fraunhofer ont-ils lieu? Est-ce
		à la surface du Soleil, entre le Soleil et la Terre, dans l'atmosphère, à la surface
		de la Terre?

\end{enumerate}


\exo{Notions de spectroscopies}

\begin{enumerate}[\bf 1)]
\item On consid\`ere l'atome d'hydrog\`ene (Z=1) et d'h\'elium (Z=2).
\begin{enumerate}
\item Expliquer pourquoi He$^+$ est un hydrog\'eno\"ide.
\item Calculer en eV les potentiels d'ionisations pour H et He$^+$.
\item Calculer les \'energies des diff\'erents niveaux quantiques ($n$ = 2 \`a 6) pour l'hydrog\`ene et He$^+$
en eV et en Joules. Tracer ces niveaux sous formes de diagrammes en eV.
\end{enumerate}

\item Lorsque l'\'electron se d\'esexcite (de $n_i$ \`a $n_f < n_i$) il \'emet de la lumi\`ere avec
une \'energie correspondant exactement \`a la diff\'erence d'\'energie entre les niveaux $n_f$ et $n_i$.
On rappelle que $E=hc/\lambda$. Le spectre d'\'emission de la s\'erie de Balmer pour l'hydrog\`ene
a \'et\'e le premier observ\'e car il est dans le domaine spectral visible (400 \`a 800 nm).
\begin{enumerate}
\item Montrer que la s\'erie de Lyman ($n_f =1$) n'est pas dans le domaine visible.
\item A quelle \'energie correspond la longueur d'onde 400 nm ?
\item A quelle \'energie correspond la longueur d'onde 800 nm ?
\item Sur la base de l'exercice 1c), donner pour He$^+$ des valeurs de $n_i$ et de $n_f$ pour
que la transition se produise dans le visible, commenter.
\end{enumerate}

%\item Bien avant la th\'eorie de Bohr, Balmer et Rydberg ont \'etabli la relation empirique
%permettant de calculer les longueurs d'onde des raies d'\'emission du spectre de l'atome d'hydrog\`ene~:
%\begin{equation}
%\label{eq_ryd}
%\frac{1}{\lambda} = R_H \left( \frac{1}{n^2_f} - \frac{1}{n^2_i} \right)
%\end{equation}
%dans laquelle $n_f$ et $n_i$ sont des entiers et $R_H$ une constante, appel\'ee constante de Rydberg.
%Trouver que $R_H = 1,097.10^7 m^{-1}$.
%%Calculer la valeur de la constante de Rydberg.
\end{enumerate}

\exo{Absorption et emission}

Example de la vie currente

%{\textcolor{red} Donn\'ees}

%\vspace{1cm}
%\small{
%\begin{tabular}{lll}
%\multicolumn{3}{l}{\bf Donn\'ees}\\
%$c$              & (vitesse de la lumi\`ere) & = 3*108 m.s$^{-1}$\\
%$h$              & (constante de Planck)     & = 6.624*10$^{-34}$ J.s\\
%$m$              & (masse de l'\'electron)   & = 9.10*10$^{-31}$ kg\\
%$e$              & (charge \'el\'ementaire)  & = 1.602*10$^{-19}$ C\\
%$\varepsilon_0$  & (permitivit\'e du vide)   & = 8.854*10$^{-12}$ C$^2$.J$^{-1}$.m$^{-1}$\\
%\multicolumn{3}{l}{1 J = 1 m$^2$kg.s$^{-2}$}\\
%\multicolumn{3}{l}{1 J = 6.241*10$^{18}$ eV}\\
%\end{tabular}
%}

