%--------------------------------------------------------------------
\titreTD{\thenumTD}{Les structures de Lewis}
%--------------------------------------------------------------------

\meth{L'algorithme de Lewis}

Rappeler les \'etapes pour construire une structure de Lewis, exemple du CO$_2$ 
(C central).
\textsl{%
\begin{enumerate}[\bf {Étape} 1]
\item Dessiner le squelette $\sigma$ de la molécule.
\item Compter le nombre total d'électrons de valence, on appelle ce nombre $n_2$.
\item Compléter la structure avec des doublets non liants pour que la règle de l'octet soit
respectée pour chaque atome.
\item Compter le nombre d'électrons dessinés. On appelle ce nombre $n_4$.
\begin{itemize}
\item si $n_4$>$n_2$ supprimer deux doublets non liants portés par deux atomes
adjacents et faire une liaison multiple entre ces atomes.
Repéter le processus jusqu'à ce que $n_4=n_2$.
\item sinon ajouter des doublets non liants sur les atomes hypervalents.
Repéter le processus jusqu'à ce que $n_4=n_2$.
\end{itemize}
\item Déterminer les charges formelles. Si deux charges opposées sont portées par deux atomes adjacents
et que l'un d'eux peut-être hypervalent, supprimer un doublet porté par l'atome chargé négativement
et faire une liaison multiple entre les deux atomes.
Repéter le processus autant que nécessaire.
\end{enumerate}}

\exo{Structure de Lewis de mol\'ecules simples}

Pour chaque atome des compos\'es suivants  (diff\'erents de H), indiquer sa position dans 
le tableau p\'eriodique (p\'eriode et groupe). Compter les \'electrons de valence
et donner la structure de Lewis de chaque compos\'e. 
\'Etablir la charge de chaque atome.
%\'Etablir la charge et le degr\'e d'oxydation de chaque atome.

\begin{center}
\begin{tabular}{lll}
\hline                                           
H$_2$O      & HCN      & PCl$_3$                   \\
CH$_4$      & CH$_3$OH & CH$_3$CH$_3$              \\
NH$_3$      & HBr      & H$_2$S                    \\
NH$_2$OH    & SiH$_4$  & HCl                       \\
NF$_3$      & PH$_3$   & BH$_3$                    \\
\hline
\end{tabular}
\end{center}

\exo{Structures de Lewis et p\'eriode 3}

%Pour les syst\`emes suivants, \'etablir les structures de Lewis, la charge et le degr\'e
%d'oxydation de chaque atome. Quel est la particularit\'e des \'el\'ements P, Cl et S dans ces compos\'es~?
Pour les syst\`emes suivants, \'etablir les structures de Lewis et la charge de chaque atome.
Quel est la particularit\'e des \'el\'ements P, Cl et S dans ces compos\'es~?

{\small \textit{Pour construire le squelette de la mol\'ecule~: l'atome central est en gras et les atomes d'hydrog\`ene 
sont connect\'es aux atomes d'oxyg\`ene.}}

\vspace{0.5cm}

%\centerline{H$_3$\textbf{P}O$_4$, H$_3$\textbf{P}O$_2$, H$_2$\textbf{S}O$_4$, H\textbf{Cl}O, H\textbf{Cl}O$_4$.}
\centerline{H$_3$\textbf{P}O$_4$, H$_2$\textbf{S}O$_4$, H\textbf{Cl}O$_4$.}

\exo{Structures de Lewis d'ions}

Pour les syst\`emes suivants, \'etablir les structures de Lewis et la charge de chaque atome.

\centerline{PF$_4^+$, CH$_3^+$, CH$_3^-$, NO$_2^-$}

%\exo{Structures de Lewis \'equivalentes} 
%
%Plusieurs structures de Lewis sont envisageables pour les mol\'ecules suivantes. 
%Les \'etablir pour chaque mol\'ecule. 
%
%\vspace{0.5cm}
%
%\centerline{NO$_2^-$, HNO$_3$, SO$_4^{2-}$, SO$_3$, HCO$_3^-$, CO$_3^{2-}$, ClO$_4^-$, O$_3$.}
%
