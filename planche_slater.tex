\titreTD{\thenumTD}{Mod\`ele de Slater}

\exo{Application du mod\`ele de Slater \`a l'h\'elium}
On se propose dans ce probl\`eme de d\'eterminer l'\'energie de premi\`ere ionisation de l'h\'elium \`a partir du mod\`ele de Slater.
\begin{enumerate}[\bf 1)]
\item Quelle est la valeur du num\'ero atomique $Z$ de l'atome d'h\'elium~?
\item Quelle est la structure \'electronique de cet \'el\'ement dans son \'etat fondamental~?
\item Donnez la valeur des nombres quantiques de chaque \'electron.
\item Sachant que l'effet d'\'ecran d'un \'electron 1$s$ sur un \'electron 1$s$ est d\'ecrit par la constante de Slater $\sigma_{1s}=0.30$, calculez la charge nucl\'eaire effective ressentie par un tel \'electron. 
\item En d\'eduire l'\'energie d'un \'electron de l'atome d'h\'elium.
\item En d\'eduire l'\'energie $E(\textrm{He})$ de l'atome d'h\'elium dans son \'etat fondamental.
\item Comment appelle-t-on un cation du type de He$^+$~?  En d\'eduire l'\'energie $E(\textrm{He}^+)$.
\item Montrez que l'\'energie de premi\`ere ionisation de l'h\'elium vaut $E_\textrm{ionis.}(\textrm{He})=24.2$~eV.
\item Quelle est la valeur minimale de la fr\'equence de la radiation que l'on doit utiliser pour ioniser 
une fois l'h\'elium~? 
\end{enumerate}
%--------------------------------------------------------------------------
\exo{Application du mod\`ele de Slater au magn\'esium}
\begin{enumerate}[\bf 1)]
\item Quelle est la configuration \'electronique du magn\'esium dans l'\'etat fondamental~?
\item D\'eterminez la charge nucl\'eaire effective puis l'\'energie de chaque \'electron.
\item \'Evaluez l'\'energie totale d'un atome de magn\'esium et d'un ion Mg$^+$.
\item En d\'eduire la valeur de l'\'energie de premi\`ere ionisation du magn\'esium.
\end{enumerate}
%--------------------------------------------------------------------------
\exo{Rayon atomique}
\begin{enumerate}[\bf 1)]
\item Donnez la configuration \'electronique des atomes de la deuxi\`eme p\'eriode du tableau p\'eriodique.
\item Compl\'etez le tableau ci-dessous. Les valeurs de $n$, $Z_{eff}$ et $r$ ne sont \`a calculer \textbf{que} pour 
la derni\`ere orbitale occup\'ee.

%\begin{tabular}{p{3cm}|c|c|c|c|c|c|c|c}
%\hline
%\'El\'ement (Z)                               & Li (3) & Be (4) & B (5) & C (6) & N (7) & O (8) & F (9) & Ne (10) \\
%\hline
%$n$ de la derni\`ere orbitale occup\'ee       &&&&&&&&\\
%$Z_{eff}$ de la derni\`ere orbitale occup\'ee & 1.3    & 1.95   &&&&&&\\
%$r$ de la derni\`ere orbitale occup\'ee       &        &        & 1.54 $a_0$ & 1.23 $a_0$ & 1.03 $a_0$ & 0.88 $a_0$& 0.77 $a_0$&\\
%\hline
%\end{tabular}

\vrule

\begin{tabular}{p{3cm}|c|c|c|c|c|c|c|c}
\hline
\textbf{\'El\'ement (Z)}                               & Li (3) & Be (4) & B (5) & C (6) & N (7) & O (8) & F (9) & Ne (10) \\
\hline
$n$        &&&&&&&&\\\hline
$Z_{eff}$  & 1.3    & 1.95   &&&&&&\\\hline
$r$        &        &        & 1.54 $a_0$ & 1.23 $a_0$ & 1.03 $a_0$ & 0.88 $a_0$& 0.77 $a_0$&\\
\hline
\end{tabular}

\vrule

\item Tracez les courbes donnant la variation de $r$ et de $Z_{eff}$ 
(de la derni\`ere orbitale occup\'ee) en fonction de Z. 
\end{enumerate}
%--------------------------------------------------------------------------
\exo{Potentiel d'ionisation}
\begin{enumerate}[\bf 1)]
\item Calculer les \'energies des atomes suivants~: C, He, Li, Ne, Al.
\item Calculer les \'energies de leurs monocations~: C$^+$, He$^+$, Li$^+$, Ne$^+$, Al$^+$.
%\marginpar{Paola: correzione PI He, Al}
\item Pour chaque esp\`ece, retouver les suivantes valeurs pour les potentiels de premi\`ere ionisation (en eV)~:
11,5 (C); 24,2 (He); 5,7 (Li); 16,0 (Ne); 10,8 (Al).
\end{enumerate}
